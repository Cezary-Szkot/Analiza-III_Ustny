\documentclass{article}

\usepackage[utf8]{inputenc}
\usepackage{geometry} \geometry{margin=70pt,tmargin=55pt, bmargin=75pt}
\usepackage[polish]{babel}
\usepackage[OT4]{fontenc}
\usepackage{polski}
\usepackage{yfonts}
\usepackage{indentfirst}
\usepackage{amsmath}
\usepackage{amsthm}
\usepackage{amsfonts}
\usepackage{amssymb} 
\usepackage{bm}
\usepackage{mathtools}
\usepackage{enumitem}
\usepackage{xcolor}
\usepackage{tcolorbox}        \tcbuselibrary{theorems}
                              \tcbuselibrary{skins}
\usepackage{tikz} 
\usepackage{hyperref}
\usepackage{graphicx}         \graphicspath{{img/}}
\usepackage{braket}
\usepackage{physics}
\usepackage{tabularx}
\usepackage{titlesec}         \titlelabel{\thetitle.\quad}
\usepackage{chngcntr}
\usepackage{titling}
\usepackage{array}            \newcolumntype{M}[1]{>{\centering\arraybackslash}m{#1}}
\usepackage{bredzenie}
\usepackage{textgreek}
\usepackage{gensymb}
\usepackage[section]{placeins}
\usepackage{float}
\usepackage{multirow}
\usepackage{makecell}
\usepackage{subcaption}
\usepackage{caption}          %\captionsetup[table]{skip=2.5pt}\captionsetup[table]{labelsep=period}
                              %\captionsetup[figure]{skip=0pt}\captionsetup[figure]{labelsep=period}

\DeclareFontFamily{U}{mathx}{\hyphenchar\font45}
\DeclareFontShape{U}{mathx}{m}{n}{
      <5> <6> <7> <8> <9> <10>
      <10.95> <12> <14.4> <17.28> <20.74> <24.88>
      mathx10
      }{}
\DeclareSymbolFont{mathx}{U}{mathx}{m}{n}
\DeclareFontSubstitution{U}{mathx}{m}{n}
\DeclareMathAccent{\widebar}{0}{mathx}{"73}

\makeatletter
\renewcommand\@biblabel[1]{#1}
\makeatother

\posttitle{\par\end{center}}
\setlength{\droptitle}{-10pt}



\counterwithout{subsection}{section} % Wyłącza resetowanie licznika podsekcji przy nowej sekcji
\renewcommand{\thesubsection}{\arabic{subsection}} % Usuwa numer sekcji z numeracji podsekcji

\titleformat{\subsection}[block] % Formatowanie nagłówka podsekcji
{\normalfont\large\bfseries}
{Pytanie \thesubsection. } % Dodaje "Pytanie X." przed tytułem
{0em}
{}




%%%%%%%%%%
% KOLORY %
%%%%%%%%%%

\definecolor{col1}{HTML}{ffcccc}
\definecolor{col2}{HTML}{ccccff}
\definecolor{darkred}{HTML}{8B0000}
\definecolor{darkblue}{HTML}{00008B}
\definecolor{goldenrod}{HTML}{FFDF42}
\definecolor{lightgoldenrod}{HTML}{fff0a5}

%%%%%%%%%
% RAMKI %
%%%%%%%%%


\newtcbtheorem[auto counter]{dfn}{Definicja}{%
  lower separated=false,
  colback=white,
  colframe=darkred,fonttitle=\bfseries,
  colbacktitle=darkred,
  coltitle=white,
  sharp corners,
  enhanced,
  attach boxed title to top left={yshift=-0.1in,xshift=0.15in},
  boxed title style={boxrule=0pt,colframe=white},
  before upper=\vspace{5pt},
  separator sign={.}
}{def}

\newtcbtheorem[auto counter]{tw}{Twierdzenie}{%
  lower separated=false,
  colback=white,
  colframe=col2,fonttitle=\bfseries,
  colbacktitle=col2,
  coltitle=black,
  sharp corners,
  enhanced,
  attach boxed title to top left={yshift=-0.1in,xshift=0.15in},
  boxed title style={boxrule=0pt,colframe=white,},
  before upper=\vspace{5pt},
  separator sign={.}
}{thm}

\newtcolorbox{wn}[2][]{%
  enhanced,colback=white,colframe=white,coltitle=black,
  sharp corners,boxrule=1.5pt,
  fonttitle=\bfseries,top=13pt,
  attach boxed title to top left={yshift=-\tcboxedtitleheight/2, xshift=0pt},
  boxed title style={tile,size=small,left=5pt,right=5pt, 
  colback=lightgoldenrod,before upper=\strut},
  title=#2,#1}


%%%%%%%%%%%
% KOMENDY %
%%%%%%%%%%%

\newcommand\R{\mathbb{R}}
\newcommand\suni{\sum_{i=1}^{n}}
\newcommand\sumi{\sum_{i=1}^{m}}
\newcommand\sunj{\sum_{j=1}^{n}}
\newcommand\sumj{\sum_{j=1}^{m}}
\newcommand\at[2]{\left.#1\right|_{#2}}
\newcommand\dz[2]{\left\langle #1,\, #2 \right\rangle}
\newcommand{\id}{\mathrm{id}}
\newcommand{\sgn}{\operatorname{sgn}}

\makeatletter
\renewcommand*\env@matrix[1][\arraystretch]{%
  \edef\arraystretch{#1}%
  \hskip -\arraycolsep
  \let\@ifnextchar\new@ifnextchar
  \array{*\c@MaxMatrixCols c}}
\makeatother



\title{\begin{Huge}Quoddam pium votum, maximatione examinis oralis ex analysi auncupatum – ad corda fovenda populo studiosorum, pro salute servanda ab infausto spectro mortis\end{Huge} \\ \begin{LARGE} (non transitus disciplinae) \end{LARGE}}

%\title{\begin{LARGE}{PEWNE ŻYCZENIE ZBOŻNE, \\ MAKSOWANIEM USTNEGO Z ANALIZY ZWANE,}\end{LARGE} \\ ku pożytkowi debilnemu ludowi studenckiemu}
\author{Marcel Tartanus}
\date{\today}




\begin{document}
\maketitle

\section*{\centering GEOMETRIA RÓŻNICZKOWA}

Odpowiedzi na pierwsze sześć pytań, których nie zdążyłem jeszcze przepisać do latexa, można znaleźć na dysku \url{https://drive.google.com/drive/folders/1cEebV2E7BEeXJyBxo_TQnKqzHyMKAuzH?usp=sharing}. W odpowiedziach do pytań, rozmaitość $M$ domyślnie jest wymiaru $n$, chyba że występuje w parze z rozmaitością $N$, wówczas $\dim{M}=m$ a $\dim{N}=n$.

\subsection{przestrzeń topologiczna, mapa, atlas klasy \texorpdfstring{$C^k$}{TEXT}, rozmaitość różniczkowa}

% \begin{dfn}[]{Przestrzeń topologiczna}{topologia}
%   Przestrzenią topologiczną nazywamy zbiór $X$ z nadaną strukturą topologiczną (\textit{topologią}) $\tau$, która to jest rodziną podzbiorów $X$, zwanych otwartymi, spełniającą następujące aksjomaty:
%   \begin{enumerate}
%     \item $\emptyset, X\in \tau$,
%     \item $\forall_{i\in I}\ A_i \in \tau \implies \bigcap\limits_{i\in I}{A_i}\in\tau$, gdzie $I$ to \underline{skończony} zbiór indeksów,
%     \item $\forall_{i\in I}\ A_i \in \tau \implies \bigcup\limits_{i\in I}{A_i}\in\tau$, gdzie $I$ to \underline{dowolny} zbiór indeksów.
%   \end{enumerate}
% \end{dfn}









\subsection{twierdzenie o powierzchniach zanurzonych}

\subsection{przestrzeń styczna do rozmaitości, wektor styczny do danej krzywej}

\subsection{baza przestrzeni stycznej związana z danym układem współrzędnych, wpływ zmiany układu współrzędnych na współczynniki wektora}

\subsection{definicja odwzorowania stycznego, transport wektorów rozmaitościami, wzór na odwzorowanie styczne w danych układach współrzędnych}  \label{sec:pchnięcie}

\subsection{wiązka styczna, pole wektorowe}










\subsection{przestrzeń kostyczna do rozmaitości, kowektory}

\begin{dfn}{Przestrzeń kostyczna}{prz.kostyczna}
  Przestrzeń dualną do przestrzeni stycznej $T_pM$, czyli $T_p^*M=L(T_pM,\,\mathbb{R})$ określamy mianem \textit{kostycznej} w punkcie $p$ do rozmaitości $M$. Jej elementami są \textit{kowektory} $\omega_p\in T_p^*M$.
\end{dfn}

\begin{dfn}{Różniczka funkcji $f:M\to\R$}{różniczka}
  Różniczką funkcji $f:M\to\mathbb{R}$ w punkcie $p$ nazywamy odwzorowanie
  \[ (\dd{f})_p: T_pM\ni v_p\mapsto v_p(f)\in\mathbb{R}. \]
  Jest ono liniowe ze względu na $v_p$, więc $(\dd{f})_p$ jest elementem przestrzeni kostycznej $T_p^*M$.
\end{dfn}

  










\subsection{baza przestrzeni kostycznej związana z danym układem współrzędnych, wpływ zmiany układu współrzędnych na współczynniki kowektora}

Mając zdefiniowaną różniczkę funkcji rzeczywistej na rozmaitości (def. \ref{def:różniczka}) możemy się teraz zastanowić nad \textbf{bazą przestrzeni kostycznej}. Weźmy sobie mapę $\phi: M \to \R^n$, taką że dla $q\in M$
\[ \phi(q) = \mqty[x_1(q) \\ \vdots \\ x_n(q)]\text{, gdzie $x_i(q)$ to $i$-ta współrzędna punktu $q$.} \]
Sprawdźmy czy różniczki funkcji $x_i: M \to \R$ stanowią bazę dualną do $\set{\at{\partialderivative{x_i}}{p}}$:

\[ (\dd{x_i})_p \at{\partialderivative{x_j}}{p} = \at{\partialderivative{x_i}{x_j}}{p} = \at{\derivative{t}(x_i\circ\gamma_j(t))}{t=0} = \at{\derivative{(t\,\delta_{ij})}{t}}{t=0} = \delta_{ij}. \]
Przedostatnia równość wynika z tego, że $\phi\circ\gamma_j = (0,\ldots, t, \ldots, 0)$. Zatem tak, zbiór różniczek $\set{(\dd{x_i})_p}$ stanowi bazę dualną do $\set{\at{\partialderivative{x_i}}{p}}$, więc jest bazą przestrzeni kostycznej $T^*_pM$.  


Jak wygląda \textbf{rozkład różniczki} $(\dd{f})_p$, czyli kowektora, w bazie $\set{(\dd{x_i})_p}$? Korzystając z dualności baz:
\[
  (\dd{f})_p \at{\pdv{x_i}}{p} 
  = \sunj \alpha_j (\dd{x_j})_p \at{\partialderivative{x_i}}{p} = \sunj \alpha_j \delta_{ji} = \alpha_i. 
\]
Z drugiej strony posiłkując się definicją \ref{def:różniczka}:
\[ 
  (\dd{f})_p \at{\pdv{x_i}}{p} = \at{\pdv{x_i}}{p} (f) = \at{\pdv{\left( f\circ\phi^{-1}(\vec{x}) \right)}{x_{i}}}{\vec{x}=\phi(p)}.
\]
Porównując ze sobą wzory otrzymujemy rozkład postaci:
\begin{equation}\label{eq:rozkład_różniczki}
(\dd{f})_p = \suni \at{\pdv{\left( f\circ\phi^{-1}(\vec{x}) \right)}{x_{i}}}{\vec{x}=\phi(p)} (\dd{x_i})_p = \suni \at{\pdv{f}{x_i}}{p} (\dd{x_i})_p,
\end{equation}
gdzie wyraz po drugiej równości oznacza uproszczoną notację. Funkcji na rozmaitości nie można różniczkować jak te na przestrzeniach euklidesowych, o ile nie obłoży się ich jakąś mapą.

Rozważmy kolejną mapę $\widetilde{\phi}$ na tej samej rozmaitości. Jaki będzie \textbf{rozkład różniczki w innej bazie} $\set{(\dd{\widetilde{x}_i})_p}$ związanej z nowym układem współrzędnych $\widetilde{\phi}$? 
\[
  (\dd{f})_p = \suni \at{\pdv{f}{x_i}}{p} \id^*(dx_i)_p = \suni \at{\pdv{f}{x_i}}{p} \sunj \at{\pdv{x_i}{\widetilde{x}_j}}{p} (\dd{\widetilde{x}_j})_p = 
  \sunj \underbrace{\left(\suni \at{\pdv{f}{x_i}}{p} \at{\pdv{x_i}{\widetilde{x}_j}}{p}\right)}_{\text{wsp. w nowej bazie}} (\dd{\widetilde{x}_j})_p,
\]
gdzie $\id^*$ to odwzorowanie kostyczne do identyczności $\id=\phi\circ\widetilde{\phi}^{-1}$. Więcej na temat odwzorowań kostycznych można znaleźć w odpowiedzi do następnego pytania. Powyższy wynik jest iloczynem kowektora (macierz $1\times n$) w starej bazie i macierzy Jacobiego przejścia między układami współrzędnych $\pdv{(x_1,\ldots x_n)}{(\widetilde{x}_1,\ldots \widetilde{x}_n)}$:
\[
\at{\mqty[\displaystyle\suni \pdv{f}{x_i} \pdv{x_i}{\widetilde{x}_1}\ \cdots\ \suni \pdv{f}{x_i} \pdv{x_i}{\widetilde{x}_n}]}{p} = \at{\mqty[\displaystyle\pdv{f}{x_1}\ \cdots\ \pdv{f}{x_n}]}{p}
 \at{\begin{bmatrix}[1.5]
   \pdv{x_1}{\widetilde{x}_1} & \pdv{x_1}{\widetilde{x}_2} & \cdots & \pdv{x_1}{\widetilde{x}_n} \\
   \pdv{x_2}{\widetilde{x}_1} & \pdv{x_2}{\widetilde{x}_2} & \cdots & \pdv{x_2}{\widetilde{x}_n} \\
   \vdots  & \vdots  & \ddots & \vdots  \\
   \pdv{x_n}{\widetilde{x}_1} & \pdv{x_n}{\widetilde{x}_2} & \cdots & \pdv{x_n}{\widetilde{x}_n}
 \end{bmatrix}}{p}.
\]









\subsection{definicja odwzorowania kostycznego, transport kowektorów między rozmaitościami, wzór na odwzorowanie kostyczne w danych układach współrzędnych} \label{sec:cofnięcie}

\begin{dfn}{Odwzorowanie kostyczne}{cofnięcie}
  Niech $M$ i $N$ będą rozmaitościami różniczkowymi klasy $C^k$, a $F:M\to N$ funkcją klasy $C^k$. Odwzorowanie $F^*: T^*_{F(p)}N\to T^*_p M$, zadane warunkiem
  \[
  \forall_{\omega_{F(p)}\in T^*_{F(p)}N}\, \forall_{v_{p}\in T_p M}\ 
  \left( F^*\omega_{F(p)} \right)(v_p) = \omega_{F(p)}\left( F_* v_p \right),
  \]
  nazywamy odwzorowaniem kostycznym do $F$, lub inaczej cofnięciem kowektorów.
\end{dfn}

W odpowiedzi na pytanie \ref{sec:pchnięcie}. pokazano, że pchnięcie wektorów możemy wyrazić za pomocą wzoru:
\[
F_*\left( \sumi v_{i} \at{\pdv{x_i}}{p} \right) = \sum_{i,\,j} [F'_{\psi\phi}]_{ji}\, v_i \at{\pdv{y_j}}{F(p)}.
\]
Posługując się nim oraz definicją odwzorowania kostycznego, można rozpisać:
\begin{align*}
  & \left( F^*\Biggl( \sunj \omega_j(F(p))\, (\dd{y_{j}})_{F(p)} \Biggr) \right) \Biggl( \sumi v_i \at{\pdv{x_i}}{p}  \Biggr) 
  =  \left( \sunj \omega_j(F(p))\, (\dd{y_j})_{F(p)} \right) \Biggl( F_*\Biggl( \sumi v_i \at{\pdv{x_i}}{p} \Biggr) \Biggr) \\
  &=  \left( \sunj \omega_j(F(p))\, (\dd{y_j})_{F(p)} \right) \Biggl( \sum_{i,\,j} [F'_{\psi\phi}]_{ji}\,v_i\at{\pdv{y_j}}{F(p)} \Biggr)  
  =  \sunj \omega_{j}(F(p)) \sumi [F'_{\psi\phi}]_{ji}\,v_i  \\
  &=  \sunj\sumi \omega_{j}(F(p))\,[F'_{\psi\phi}]_{ji}\,v_i
  = \sunj\sumi \omega_{j}(F(p))\,[F'_{\psi\phi}]_{ji}\, \bigl( \dd{x_i} \bigr)_p (v_p)
  = \sunj\sumi \bigl([F'_{\psi\phi}]^T\bigr)_{ij}\, \omega_{j}(F(p)) \bigl( \dd{x_i} \bigr)_p (v_p)
\end{align*}
Uzyskujemy zatem \textbf{wzór na odwzorowanie kostyczne} w układach współrzędnych (mapach) $\psi$ i $\phi$ na rozmaitościach odpowiednio $N$ i $M$:
\begin{equation*}
  \color{darkred} F^* \omega_{F(p)} = \sunj\sumi \bigl([F'_{\psi\phi}]^T\bigr)_{ij}\, \omega_{j}(F(p)) \bigl( \dd{x_i} \bigr)_p. 
\end{equation*}
Widzimy więc, że cofnięcie kowektorów jest operacją liniową o macierzy $[F'_{\psi\phi}]^T$.








\subsection{wiązka kostyczna, 1-forma, k-forma}

\begin{dfn}{Wiązka kostyczna}{wiązka_kostyczna}
  Wiązka kostyczna to suma rozłączna przestrzeni kostycznych w każdym punkcie rozmaitości
  \[
  T^*M = \bigcup_{p\in M} \set{p}\times T^*_pM = \set{(p, \omega)| p\in M,\, \omega\in T^*_p M}.
  \]
\end{dfn}

\begin{dfn}{Pole kowektorowe (1-forma różniczkowa)}{1-forma}
  Pole kowektorowe jest funkcją $\omega:M\to T^*M$, taką że 
  \[
  M\ni p\mapsto \omega_p = \suni (\omega_p)_i (\dd{x_i})_p \in T_p^*M.
  \]
  Traktując $(\omega_p)_i$ jako odwzorowanie na $M$ oraz wprowadzając $\omega_i: M\ni p \mapsto (\omega_p)_i\in\R$, pole $\omega$ można zapisać jako:
  \[
  \omega(p) = \suni \omega_i(p) (\dd{x_i})_p.
  \]
  Jednoformami można \textit{działać na pola wektorowe} $V\in\mathfrak{X}(M)$, przez co rozumiemy: $\at{\omega(V)}{p} = \omega_p(v_p)$.
  Wynikiem takiej operacji jest funkcja $\omega(V):M\to \R.$
\end{dfn}

Jak wygląda 1-forma funkcji rzeczywistej na rozmaitości? Jest to odwzorowanie $\dd{f}: p\mapsto (\dd{f})_p$, gdzie $(\dd{f})_p$ to różniczka funkcji $f$ w punkcie $p$ o rozkładzie w bazie $\set{(\dd{x_i})_p}$ (\ref{eq:rozkład_różniczki}). Należy tutaj poczynić pewną \textit{uwagę}. Dla każdej funkcji $f$ istnieje 1-forma $\omega$, taka że $\omega=\dd{f}$, ale nie dla każdej 1-formy istnieje funkcja z różniczką odpowiadającą 1-formie. 


\begin{dfn}{k-forma różniczkowa}{k-forma}
  Niech $\Lambda^k T_pM = \Lambda^k_pM$ oznacza przestrzeń liniową całkowicie antysymetrycznych $k$-liniowych funkcjonałów na $(T_pM)^k=T_pM\times\cdots\times T_pM$. Formą różniczkową nazwiemy funkcję $\omega: M\to \bigcup_{p\in M} \Lambda^k_p M$, taką że $\omega(p)\in \Lambda^k_p M$. Zbiór wszystkich $k$-form różniczkowych na $M$ oznaczamy jako $\Lambda^k M$. Lokalnie, tj. w otoczeniu ustalonego punktu $p$ rozmaitości, w dziedzinie pewnego układu współrzędnych $\phi$, dowolną $k$-formę $\omega$ można przedstawić jednoznacznie w postaci:
  \[
  \omega = \sum_{1\leq i_1 < \cdots < i_k \leq n} \omega_{i_1,\ldots, i_k}(\vec{x}) \dd{x}_{i_1}\wedge\cdots\wedge\dd{x}_{i_k} 
  = \frac{1}{k!}\,\omega_{i_1,\ldots, i_k}(\vec{x})\dd{x}^{i_1}\wedge\cdots\wedge\dd{x}^{i_k},
  \]
  gdzie $\omega_{i_1,\ldots, i_k}$ jest antysymetrycznym współczynnikiem a $\vec{x}=\phi(p)$. W drugiej równości zastosowano konwencję sumacyjną Einsteina.
\end{dfn}
W tym miejscu dodefiniujemu, że $\Lambda^0_p M=\mathbb{R}$. Bazą przestrzeni $\Lambda^k_p M$ jest zbiór iloczynów zewnętrznych kowektorów $\set{\dd{x_{i_1}}\wedge\cdots\wedge\dd{x_{i_k}}}_{1\leq i_1<\cdots<i_k\leq n}$, więc jej wymiar wynosi $\binom{n}{k}$. Warto zauważyć, iż $\dim{\Lambda^k_p M}=\dim{\Lambda^{n-k}_p M}$.

W powyższej definicji pisałem o antysymetrycznych $k$-liniowych funkcjonałach. Pod tym hasłem kryją się antysymetryczne $k$-formy rozumiane w szerszym sensie, tak jak na Algebrze I. Nie użyłem tego sformułowania, by uniknąć ich poplątania z formami różniczkowymi przez czytelnika. Od tej chwili notacja sumacyjna Einsteina będzie się coraz częściej pojawiała. Została ona dobrze wytłumaczona w skrypcie Alatosa (\url{https://www.fuw.edu.pl/~alatos/analiza-wyklady.pdf} str. 16 -- 22), więc nie poświęcimy jej tu więcej miejsca.













\subsection{iloczyn zewnętrzny, iloczyn wewnętrzny, różniczka zewnętrzna} \label{sec:iloczyny_różniczka}

\begin{dfn}{Iloczyn zewnętrzny}{iloczyn_zew}
  Niech $V$ będzie przestrzenią wektorową. Przez iloczyn zewnętrzny wektorów $v_1,\ldots,v_k\in V$ rozumiemy
  \[
  v_1\wedge\cdots\wedge v_k = \sum_{\sigma\in S_k} \sgn(\sigma)\, v_{\sigma(1)}\otimes\cdots\otimes v_{\sigma(k)}. 
  \]
  Przestrzenią liniową jest również $T^*_p M$, więc rozważać można iloczyn zewnętrzny kowektorów. W takim przypadku dla $\omega_1,\ldots,\omega_k\in T^*_p M$ oraz $v_1,\ldots,v_k\in T_p M$:
  \[
  \Bigl( \omega_1\wedge\cdots\wedge\omega_k \Bigr)(v_1\wedge\cdots\wedge v_k) = \sum_{\sigma\in S_k} \sgn(\sigma)\, \omega_{\sigma(1)}(v_1)\cdots \omega_{\sigma(k)}(v_k) \in \Lambda^k_p M.
  \]
  Wśród własności iloczynu zewnętrznego można wymienić:
  \begin{itemize}
    \item \textbf{antysymetryczność} $\omega_1\wedge\cdots\wedge\omega_i\wedge\cdots\wedge\omega_j\wedge\cdots\wedge\omega_k = -\omega_1\wedge\cdots\wedge\omega_j\wedge\cdots\wedge\omega_i\wedge\cdots\wedge\omega_k$,
    \item \textbf{wieloliniowość} $\omega_1\wedge\cdots\wedge\,(\alpha\omega_i+\beta\widetilde{\omega}_i)\,\wedge\cdots\wedge\omega_k = \alpha(\omega_1\wedge\cdots\wedge\omega_i\wedge\cdots\wedge\omega_k) + \beta(\omega_1\wedge\cdots\wedge\widetilde{\omega}_i\wedge\cdots\wedge\omega_k)$,
    \item \textbf{łączność} $(\omega_1\wedge\cdots\wedge\omega_k)\wedge(\widetilde{\omega}_1\wedge\cdots\wedge\widetilde{\omega}_n)=\omega_1\wedge\cdots\wedge\omega_k\wedge\widetilde{\omega}_1\wedge\cdots\wedge\widetilde{\omega}_n$.
  \end{itemize}
\end{dfn}
Więcej informacji o tensorach i iloczynie tensorowym, który pojawił się w powyższej definicji, można przeczytać w moich notatkach z ćwiczeń ze Zglinickim \url{https://drive.google.com/file/d/16qXaC4W0_puQSz4xwP2QBLQm9_V_2kir/view?usp=sharing}.


\begin{dfn}{Iloczyn wewnętrzny (skalarny)}{iloczyn_wew}
  Niech $V$ będzie przestrzenią wektorową nad ciałem $\mathbb{K}$ ($\R$ lub $\mathbb{C}$). Odwzorowanie $V\times V\ni(v,w)\to\braket{v}{w}\in\mathbb{K}$ określamy \textit{iloczynem wewnętrznym} (\textit{skalarnym}), jeśli dla $v, w, w_1, w_2\in V$ oraz $\lambda_1, \lambda_2\in\mathbb{K}$ mamy:
  \begin{enumerate}
    \item $\braket{v}{\lambda_1 w_1 + \lambda_2 w_2} = \lambda_1\braket{v}{w_1} + \lambda_2\braket{v}{w_2}$ (liniowość w drugim argumencie),
    \item dla $\mathbb{K}=\R$:  $\braket{v}{w}=\braket{w}{v}$ (symetria),
          dla $\mathbb{K}=\mathbb{C}$: $\braket{v}{w}=\overline{\braket{w}{v}}$ (hermitowskość),
    \item $\braket{v}{v}>0$ dla $v\neq\vec{0}$ (dodatniość).
  \end{enumerate}
  Dla $\mathbb{K}=\R$ iloczyn skalarny sprowadza się do dodatnio-określonej, symetrycznej formy dwuliniowej 
\end{dfn}

















\subsection{tensor metryczny, forma objętości}\label{sec:tensor_metryczny_vol}


\begin{dfn}{Tensor metryczny}{tensor_metryczny}
  Tensorem metrycznym na rozmaitości różniczkowej $M$ zwiemy odwzorowanie $g:M\to T^*M\otimes T^*M$, takie że $M\ni p\mapsto g_p\in T^*M\otimes T^*M$, gdzie $g_p$ jest 
  \begin{itemize}
    \item \textbf{symetryczną} $\forall_{v, w\in T_pM}\ g_p(v,w) = g_p(w,v)$,
    \item \textbf{dodatnio-określoną} $\forall_{v\in T_pM\setminus\set{\vec{0}}}\ g_p(v,v)>0$,
    \item \textbf{niezdegenerowaną} $\forall_{w\in T_pM}:\ (\forall_{v\in T_pM}\ g_p(v,w)=0)\implies w=\vec{0}$,
  \end{itemize}
  dwuliniową formą na $T_pM\times T_pM$. Innymi słowy $g$ jest cięciem wiązki tensorowej $T^*M\otimes T^*M$, które zadaje \textit{lokalnie}, w punkcie $p$, iloczyn skalarny $g_p$. We współrzędnych lokalnych $(x^1,\ldots,x^n)$ tensor metryczny można zapisać jako:
  \[
  g = \sum_{i,j=1}^{n}g_{ij}(\vec{x})\dd{x^i}\otimes\dd{x^j}.
  \]
\end{dfn}
Ze względu na dodatniość dwuformy $g_p$ warunek niezdegenerowania jest zbędny. Podaje się go jednak często, bo w niektórych dziedzinach rozważa się niekoniecznie dodatnio-określone iloczyny skalarne (np. w OTW).
Współczynniki tensora metrycznego zapisujemy z dolnymi indeksami. Wyrazy postaci $g^{ij}$ rozumiemy jako współczynniki macierzy odwrotnej do $[g_{ij}]$, więc $g^{ik}g_{kj}=\delta^i_j$. Niech \( [g_{ij}] \) będzie macierzą tensora metrycznego \( g \) w bazie związanej z układem współrzędnych \( (x^i) \), a \( [\widetilde{g}_{ab}] \) macierzą tego tensora w bazie związanej z układem współrzędnych \( (y^a) \). Przejście między nimi odbywa się za pomocą macierzy Jacobiego  $J = \pdv{(x^1,\ldots,x^n)}{(y^1,\ldots,y^n)}$:  
\[
[\widetilde{g}_{ab}] = J^T [g_{ij}] J.
\]

\begin{dfn}{Forma objętości}{vol}
  Formą objętości na rozmaitości różniczkowej $M$ nazywamy $n$-formę różniczkową postaci:
  \[
  vol = \pm\sqrt{\det g}\ \dd{x^1}\wedge\cdots\wedge\dd{x^n}.
  \]
  Znak formy zależy od orientacji układu współrzędnych, w którym została wyrażona.
\end{dfn}














\subsection{Co to znaczy, że odwzorowanie zachowuje orientację rozmaitości?}{orientacja}













\end{document}


